\documentclass[
	12pt,				% tamanho da fonte
	%openright,			% capítulos começam em pág ímpar (insere página vazia caso preciso)
	%twoside,			% para impressão em recto e verso. Oposto a oneside
	openany,			%Para nao pular folhas quando um paragrafo novo começa. Oposto de Twoside e openright
	a4paper,			% tamanho do papel.
	chapter=TITLE,		% títulos de capítulos convertidos em letras maiúsculas
	section=TITLE,		% títulos de seções convertidos em letras maiúsculas
	%subsection=TITLE,	% títulos de subseções convertidos em letras maiúsculas
	%subsubsection=TITLE,% títulos de subsubseções convertidos em letras maiúsculas
	english,
	brazil				% o último idioma é o principal do documento
]{abntex2}
\usepackage[brazil]{babel}
\usepackage[utf8]{inputenc} %Pacote de linguas
\usepackage[normalem]{ulem}
\usepackage[T1]{fontenc}
\usepackage{lipsum}
\usepackage{cmap}
\usepackage{graphicx}
\usepackage[brazilian,hyperpageref]{backref}
\usepackage[alf]{abntex2cite} % Citações padrão ABNT
\usepackage{rotating}
\usepackage{float}
\usepackage{color}
\usepackage{listings}    
\usepackage{inconsolata}

\usepackage{listings}

\title{Teste de software}
\date{\today}
\autor{Felipe Menino Carlos}

\setlength{\parindent}{1.3cm}
\frenchspacing

% Adicionando idioma
\selectlanguage{brazil}

\begin{document}
\maketitle

\chapter{Exercícios}

Abaixo será listado os exercícios propostos pelo professor Arley, na segunda lista de exercícios da matéria de testes de software

\section{Exercício 6}

 Por que a frase a seguir está errada? “Uma classe parametrizada pode ser usada para otimizar qualquer conjunto de teste”

R: Esta frase está errada pois uma classe parâmetrizada não pode ser usada em todos os casos, isso porque os parâmetros que  serão testados podem não corresponder ao que é necessário para validar certos testes.

\section{Exercício 7}

O que é uma Suite de testes ?

R: É um conjunto de testes que facilita o gerênciamento dos muitos testes possíveis em um projeto, isso porque com suites de testes, eles poderão ser segmentados por partes do projeto e com isso aumentar a organização dos projetos.

\end{document}
